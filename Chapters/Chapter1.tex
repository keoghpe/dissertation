% Chapter 1

\chapter{Introduction} % Main chapter title

\label{Chapter1} % For referencing the chapter elsewhere, use \ref{Chapter1}

%----------------------------------------------------------------------------------------
%	PROJECT BACKGROUND
%----------------------------------------------------------------------------------------

\lhead{Chapter 1. \emph{Introduction}} % This is for the header on each page - perhaps a shortened title

This chapter introduces the research topic, argumentation theory and outlines it's potential as an alternative to machine learning in the implementation of decision support systems.
The aims and reasoning of this research study are clarified.
The research methodology is discussed along with the scope and limitations of the research study.
Finally the organisation of this document is presented to the reader.

%----------------------------------------------------------------------------------------
%	PROJECT BACKGROUND
%----------------------------------------------------------------------------------------

\section{Project Background}

This study occurs in the context  among the intersection of several subdomains of artificial intelligence, most notably argumentation theory and machine learning.

A basic requirement of decision support systems is to aid users by presenting results based on that data. Making predictions.

Arthur Samuel defined machine learning as the ``field of study that gives computers the ability to learn without being explicitly programmed.''
The focus on this research is on supervised machine learning; learning that makes predications based on a labeled training set of data. The predictive ability of several different.

An improvement on this basic functionality is to present the user with information on how the result was generated so that the user can factor this into their decision making process.
Machine learning techniques can provide reasonable answers to questions based on data, however, they often fall short in presenting their process to the user.
The literature in the domain suggests that argumentation theory can perform as well as, or in some cases perform better than, machine learning algorithms in predicting outcomes.
Argumentation systems have many other characteristics that make their use in decision support systems advantageous. Argumentation systems can reason based on incomplete or corrupt data and explain how an algorithm arrives at a conclusion; these are features lacking in many machine learning based systems.


%----------------------------------------------------------------------------------------
%	RESEARCH AIMS AND OBJECTIVES
%----------------------------------------------------------------------------------------

\section{Research Aims and Objectives}

The overall aim of this research is to compare the predictive ability of algorithms based on argumentation theory with machine learning ones.
Demonstrating that argumentation based approaches can perform as well as, and in some cases better than, learning based approaches opens the door to implementing these approaches in real-world decision support systems.
The objectives of this study can be summarised as follows:

\begin{itemize}

  \item To compare the performance and predictive ability of AT with ML algorithms for regression.
  \item To compare the performance and predictive ability AT with ML algorithms for regression.
  \item To provide a discussion of the benefits and drawbacks of each approach in decision support systems.

\end{itemize}

%-----------------------------------
%	RESEARCH METHODS
%-----------------------------------

\section{Research Methods}

In order to achieve the aforementioned aims the following experiments are conducted.

\begin{itemize}

  \item An argumentation system is developed in software and used to elicit a knowledge base from an expert.
  \item A number of ML classifiers (using both classification and regression algorithms) are trained to using a partition of the experiment dataset.
  \item The predictive ability of the classifiers and the knowledge based system are tested using a subset of tuples of the dataset.

\end{itemize}

%----------------------------------------------------------------------------------------
%	SCOPE AND LIMITATIONS
%----------------------------------------------------------------------------------------

\section{Scope and Limitations}

In this study we are restricted by several factors in the analysis of the techniques.
There are several limiting factors in the experiments that are a barrier to the accurate assessment of argumentation theory.
The argumentation system is limited in it's predictive capacity by it's knowledge based approach.
The system elicits a knowledge based from an ``expert'' and assumes that the expert's knowledge accurately reflects reality.
There is no objective verification of that knowledge base. The performance of the argumentation system is based heavily on the knowledge of the expert and they ability of the system to model that knowledge accurately.
We are similarly limited in our evaluation of the classifiers produced using the machine learning approach.
Obtaining a dataset of sufficient size to train classifiers to a level on par with the knowledge of the expert.
Finding a large enough dataset that reflects the knowledge of the expert is a challenge.

%-----------------------------------
%	ORGANISATION OF THESIS
%-----------------------------------

\section{Organisation of Thesis}

This thesis is organised into the following sections:

\begin{itemize}

  \item Chapter 2 provides a review of the literature relevant to argumentation theory and machine learning including cutting-edge research in the area.
  \item Chapter 3 outlines the design of the experiment and the justification of those design decisions in the context of previous research. methodology
  \item Chapter 4 details the implementation of the experiments; the implementation relevant software artifacts, their use and the overall execution of the experiments.
  \item Chapter 5 the results of the experiments are presented and discussed.
  \item Chapter 6 reviews the experiment and findings in the context of the relevant literature. Conclusions are then derived and future work suggested.

\end{itemize}
